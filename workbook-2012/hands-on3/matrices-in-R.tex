
\section{Matrices in R}

In R matrices are two-dimensional collections of elements all of which
have the same mode or type. This is different than a data frame in which
the columns of the frame can hold elements of different type (but all of
the same length), or from a list which can hold objects of arbitrary
type and length. Matrices are more efficient for carrying out most
numerical operations, so if you're working with a very large data set
that is amenable to representation by a matrix you should consider using
this data structure.

\subsection{Creating matrices in R}

There are a number of different ways to create matrices in R. For
creating small matrices at the command line you can use the
\lstinline!matrix()! function.

\begin{R}
> X <- matrix(1:5)
> X
      [,1]
 [1,]    1
 [2,]    2
 [3,]    3
 [4,]    4
 [5,]    5
> X <- matrix(1:12, nrow=4)
> X
     [,1] [,2] [,3]
[1,]    1    5    9
[2,]    2    6   10
[3,]    3    7   11
[4,]    4    8   12
> dim(X) # give the shape of the matrix 
[1] 4 3
\end{R}
\lstinline!matrix()! takes a data vector as input and the shape of the
matrix to be created is specified by using the \lstinline!nrow! and
\lstinline!ncol! arguments (if the number of elements in the input data
vector is less than |nrows| $\times$ |ncols| the
elements will be 'recycled' as discussed in previous lectures). Without
any shape arguments the \lstinline!matrix()! function will create a
column vector as shown above. By default the \lstinline!matrix()!
function fills in the matrix in a column-wise fashion. To fill in the
matrix in a row-wise fashion use the argument \lstinline!byrow=T!.

If you have a pre-existing data set in a list or data frame you can use
the \lstinline!as.matrix()! function to convert it to a matrix.

\begin{R}
> turtles <- read.table('turtles.txt', header=T)
> tmtx <- as.matrix(turtles) 
> tmtx   # note how the elements were all converted to character 
   sex length width height
1  "f" " 98"  " 81" "38"  
2  "f" "103"  " 84" "38"  
3  "f" "103"  " 86" "42"  
4  "f" "105"  " 86" "40"  
 ... output truncated ...
> tsub <- subset(turtles, select=-sex)
> tmtx <- as.matrix(tsub)
> tmtx    # this is probably more along the lines of what you want
   length width height
1      98    81     38
2     103    84     38
3     103    86     42
4     105    86     40
 ... output truncated ...
\end{R}
You can use the various indexing operations to get particular rows,
columns, or elements. Here are some examples:

\begin{R}
> X <- matrix(1:12, nrow=4)
> X
     [,1] [,2] [,3]
[1,]    1    5    9
[2,]    2    6   10
[3,]    3    7   11
[4,]    4    8   12
> X[1,] # get the first row
[1] 1 5 9
> X[,1] # get the first column
[1] 1 2 3 4
> X[1:2,] # get the first two rows
     [,1] [,2] [,3]
[1,]    1    5    9
[2,]    2    6   10
> X[,2:3] # get the second and third columns
     [,1] [,2]
[1,]    5    9
[2,]    6   10
[3,]    7   11
[4,]    8   12
> Y <- matrix(1:12, byrow=T, nrow=4)
> Y
     [,1] [,2] [,3]
[1,]    1    2    3
[2,]    4    5    6
[3,]    7    8    9
[4,]   10   11   12
> Y[4] # see explanation below
[1] 10 
> Y[5]
[1] 2
> dim(Y) <- c(2,6)
> Y
     [,1] [,2] [,3] [,4] [,5] [,6]
[1,]    1    7    2    8    3    9
[2,]    4   10    5   11    6   12
> Y[5]
[1] 2
\end{R}
The example above where we create a matrix \lstinline!Y! is meant to
show that matrices are stored internally in a column wise fashion (think
of the columns stacked one atop the other), regardless of whether we use
the \lstinline!byrow=T! argument. Therefore using single indices returns
the elements with respect to this arrangement. Note also the use of
assignment operator in conjuction with the \lstinline!dim()! function to
reshape the matrix. Despite the reshaping, the internal representation
in memory hasn't changed so \lstinline!Y[5]! still gives the same
element.

You can use the \lstinline!diag()! function to get the diagonal of a
matrix or to create a diagonal matrix as show below:

\begin{R}
> Z <- matrix(rnorm(16), ncol=4)
> Z
           [,1]       [,2]         [,3]        [,4]
[1,] -1.7666373  2.1353032 -0.903786375 -0.70527447
[2,] -0.9129580  1.1873620  0.002903752  0.51174408
[3,] -1.5694273 -0.5670293 -0.883259848  0.05694691
[4,]  0.9903785 -1.6138958  0.408543336  2.39152400
> diag(Z)
[1] -1.7666373  1.1873620 -0.8832598  2.3915240
> diag(5) # create the 5 x 5 identity matrix
     [,1] [,2] [,3] [,4] [,5]
[1,]    1    0    0    0    0
[2,]    0    1    0    0    0
[3,]    0    0    1    0    0
[4,]    0    0    0    1    0
[5,]    0    0    0    0    1
> s <- sqrt(10:13)
> diag(s)
         [,1]     [,2]     [,3]     [,4]
[1,] 3.162278 0.000000 0.000000 0.000000
[2,] 0.000000 3.316625 0.000000 0.000000
[3,] 0.000000 0.000000 3.464102 0.000000
[4,] 0.000000 0.000000 0.000000 3.605551
\end{R}

Note that the |rnorm()| function generates random numbers from the standard normal distribution. Use the help to read the documentation for |rnorm()|. Note that you can use the |mean| and |sd| arguments to specify other normal distributions.  Since we've introduced the |rnorm()| function let's go ahead and show how we can useit to simulate draws from a random normal distribution.
%
\begin{R}
> x <- rnorm(100) # draw 100 samples from random normal distn
> mean(x)
[1] 0.03198427
> sd(x)
[1] 1.012966
> hist(x)
> abline(v=mean(x),col='red',lwd=2,lty='dashed')
\end{R}
%
Also notice that the |rnorm()| help file also mentions three other related functions |dnorm()|, |pnorm()|, and |qnorm()|. |dnorm()| gives the density, |pnorm()| the distribution function, and |qnorm()| the quantile function. Here's an example how we can use the |dnorm()| function to compare our observed sample to the expected distribution:
%
\begin{R}
> breakpts <- seq(-3, 3, 0.5)
# note use of freq=F to get density histogram  and user specified breakpoints
> h <- hist(x, breakpts, freq=F)
> expected <- dnorm(h$mids)
> lines(h$mids, expected, col='blue',lwd=2)
> abline(v=0, col='blue', lwd=2)
> abline(v=mean(x), col='red', lwd=2,lty='dashed')
\end{R}
%

\subsubsection{Matrix operations in R}

The standard mathematical operations of addition and subtraction and
scalar multiplication work element-wise for matrices in the same way as
they did for vectors. Matrix multiplication uses the operator
\lstinline!%*%! which you saw last week for the dot product. To get the
transpose of a matrix use the function \lstinline!t()!. The
\lstinline!solve()! function can be used to get the inverse of a matrix
(assuming it's non-singular) or to solve a set of linear equations.

\begin{R}
> A <- matrix(1:12, nrow=4)
> A <- matrix(1:12, nrow=4)
> A
     [,1] [,2] [,3]
[1,]    1    5    9
[2,]    2    6   10
[3,]    3    7   11
[4,]    4    8   12
> t(A)
     [,1] [,2] [,3] [,4]
[1,]    1    2    3    4
[2,]    5    6    7    8
[3,]    9   10   11   12
> B <- matrix(rnorm(12), nrow=4)
> B
           [,1]        [,2]        [,3]
[1,] -2.9143953  0.38204730 -1.33207235
[2,]  0.1778266 -0.44563686  0.76143612
[3,]  1.7226235  0.03320553 -0.06652767
[4,]  0.5291281 -0.13145408  0.14108766
> A + B
          [,1]     [,2]      [,3]
[1,] -1.914395 5.382047  7.667928
[2,]  2.177827 5.554363 10.761436
[3,]  4.722623 7.033206 10.933472
[4,]  4.529128 7.868546 12.141088
> A - B
         [,1]     [,2]      [,3]
[1,] 3.914395 4.617953 10.332072
[2,] 1.822173 6.445637  9.238564
[3,] 1.277377 6.966794 11.066528
[4,] 3.470872 8.131454 11.858912
> 5 * A
     [,1] [,2] [,3]
[1,]    5   25   45
[2,]   10   30   50
[3,]   15   35   55
[4,]   20   40   60
> A %*% B  # do you understand why this generated an error?
Error in A %*% B : non-conformable arguments
> A %*% t(B)
          [,1]     [,2]     [,3]     [,4]
[1,] -12.99281 4.802567 1.289902 1.141647
[2,] -16.85723 5.296193 2.979203 1.680408
[3,] -20.72165 5.789819 4.668505 2.219170
[4,] -24.58607 6.283445 6.357806 2.757932
> C <- matrix(1:16, nrow=4)
> solve(C)  # not all square matrices are invertible!
Error in solve.default(C) : Lapack routine dgesv: system is exactly singular
> C <- matrix(rnorm(16), nrow=4)  # you'll get
> C
           [,1]       [,2]       [,3]       [,4]
[1,] -1.6920758 -0.8104245  0.9940420  0.3592050
[2,]  1.5949448 -0.9508142 -0.1960434 -0.5678855
[3,] -1.2443831  0.6400100  0.2645679 -0.8733987
[4,]  0.2129116  0.6719323  0.7494698 -0.3856085
> Cinv <- solve(C)  # this should return something that looks like an identity matrix
> C %*% Cinv
             [,1]          [,2]          [,3]          [,4]
[1,] 1.000000e+00 -2.360850e-17  6.193505e-17  4.189425e-18
[2,] 2.710844e-17  1.000000e+00  3.577867e-18 -7.264493e-17
[3,] 4.944640e-17  7.643625e-17  1.000000e+00  5.134714e-17
[4,] 1.978161e-17 -1.187201e-17 -4.022390e-17  1.000000e+00
> all.equal(C %*% Cinv, diag(4)) # test approximately equality 
[1] TRUE
\end{R}


We expect that $CC^{-1}$ should return the above should return the
$4 \times 4$ identity matrix. As shown above this is true up to the
approximate floating point precision of the machine you're operating on.



