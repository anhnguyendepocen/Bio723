

\section{The Reshape and Plyr packages}

|reshape2| and |plyr| are two powerful R packages for restructuring, subsetting, transforming, and summarizing large data sets. Both packages were written by Hadley Wickham, a statistician at Rice University, who is also the author of ggplot.  In this chapter we'll just scratch the surface of these packages; for more detailed overviews see the documentation available on their respective web pages -- \href{http://plyr.had.co.nz/}{plyr} and \href{http://had.co.nz/reshape/}{reshape}. Install the |reshape2| and |plyr| packages before proceeding.


\subsection{Yeast NanoString Dataset}

To illustrate the use of these packages we will use a gene expression data set my lab generated.  This data set includes time series expression measurements on 192 genes, collected on each of four different yeast strains grown under two different media conditions. Each treatment (time point, media condition, strain) was replicated three times.  

Download the data set |yeast-timeseries.csv| from the course wiki. The data file is a plain text file that uses the "comma separated values" format. Use the |read.csv| function to read this data into R.
%
\begin{R}
> yeast.time <- read.csv('yeast-timeseries.csv')
> dim(yeast.time)
[1] 108 196
> names(yeast.time)
  [1] "sample.id"  "media"      "strain"     "time.pt"    "replicate"  "ACE2"      
  [7] "ACT1"       "ADR1"       "AGA2"       "AMN1"       "ASG7"       "ASH1" 
...
\end{R}

The first five columns are information about the experimental design.  These are fixed variates that we want to treat as factors. Therefore, let's ensure that they are:
%
\begin{R}
yeast.time[1:5] <- lapply(yeast.time[1:5], as.factor)
\end{R}
%
Note the use of the |lapply| function to apply to a function (|as.factor| in this case) repeatedly to the elements of a list.

\subsection{Reshape package}

The |reshape2| package allows us to restructure and aggregate data more easily than the built-in R functions.  There are two primary functions associated with the package -- |melt()| and |cast()|.  We use |melt()| to restructure a data frame or list into a generic structure that can then be |cast()| into the form we want.


\subsubsection{melt}

Import the |reshape2| package with |library(reshape2)| and read the docs for the |melt()| function.  |melt()| needs at least three arguments: 1) a data frame or list, 2) a vector specifying which columns to treat as `identification variables' (|id.vars|), and 3) a vector specifying which columns to use as `measured variables' (|measured.vars|).  ID variables are typically the fixed variables that represent aspects of the experimental design while measured variables represent the variables that were measured on each unit of interest.  If |id.vars| and |measured.vars| aren't specified, the |melt()| function will try and infer the |id.vars| based on those columns that are factors, and treat the remaining variables as |measured.vars|.  If only |id.vars| is specified, the remaining variables will be treated as |meaured.vars|.

Let's create a simple data set that we can use to explore |melt()| and |cast()| functions.
%
\begin{R}
> group1 <- c(rep("A", 9), rep("B",9))
> group2 <- rep(c("1","2","3"),6)
> data1 <- c(rnorm(9,mean=0), rnorm(9,mean=1))
> data2 <- as.vector(t(mapply(rnorm, n = c(3,3,3,3,3,3), mean = c(0,1,3))))
> test.data <- data.frame(species=as.factor(group1), 
                          treatment=as.factor(group2), 
                          v1=data1, v2=data2)
> test.data
   species treatment          v1          v2
1        A         1  0.75357665 -1.83527194
2        A         2  0.40335481  1.22663973
3        A         3  1.18084161  4.47310654
4        A         1 -0.18393749 -1.61953719
5        A         2 -0.85328571  2.75230914
6        A         3  0.99141392  3.39575430
7        A         1 -1.12026845 -0.61442409
8        A         2 -0.01716075  2.08970130
9        A         3 -1.84389967  2.08822132
10       B         1 -0.30507545 -0.01171179
11       B         2  0.54634457 -0.31179921
12       B         3  2.38310469  4.02453740
13       B         1  0.95729799  0.14273026
14       B         2  3.14992630  1.01718329
15       B         3  1.28400301  2.16849192
16       B         1  0.94616082 -0.26436515
17       B         2  1.19047574  0.58964302
18       B         3  0.35085358  2.46990925
\end{R}
%

Let's apply |melt()|, specifying the `species' and `treatment' columns as the |id.vars|.
%
\begin{R}
> melt.test <- melt(test.data, id.vars = c("species","treatment"))
> melt.test
   species treatment variable       value
1        A         1       v1  0.75357665
2        A         2       v1  0.40335481
3        A         3       v1  1.18084161
....
13       B         1       v1  0.95729799
14       B         2       v1  3.14992630
15       B         3       v1  1.28400301
....
19       A         1       v2  1.91999253
20       A         2       v2  1.76509214
21       A         3       v2  3.33803728
....
28       B         1       v2  3.63924057
29       B         2       v2  1.81988816
30       B         3       v2  2.00163369
\end{R}
%
Examining the melted data set, you'll see that the columns representing the measured variables have been collapsed into a single new column called `value'.  There is also another column called `variable' which specifies which of the measured variables the items in `value' came from.

\subsubsection{cast}

Having melted our data set, we can then use the |cast()| function to reshape and aggregate the data into the form we desire.  Read the docs for |cast()|.  Note that the |cast()| function is actually caleld as |dcast()| or |acast()| depending on whether you want the function to return a data frame or a vector or array. Minimally, |cast()| takes: 1) a melted data set,  2) a formula specifying how to shape the melted data; and 3) a function to apply to any aggregates that are specified for the cast formula. These are most easily illustrated by example, as shown below.
%
\begin{R}
> recast.test <- dcast(melt.test, species ~ variable, mean)
> recast.test
  species          v1       v2
1       A -0.07659612 1.328500
2       B  1.16701014 1.091624

> recast.test <- dcast(melt.test, species + treatment ~ variable, mean) 
> recast.test
  species treatment         v1          v2
1       A         1 -0.1835431 -1.35641107
2       A         2 -0.1556972  2.02288339
3       A         3  0.1094520  3.31902738
4       B         1  0.5327945 -0.04444889
5       B         2  1.6289155  0.43167570
6       B         3  1.3393204  2.88764619   
\end{R}
