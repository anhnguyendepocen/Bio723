% Created 2013-08-21 Wed 17:11
\documentclass[11pt]{article}
\usepackage[utf8]{inputenc}
\usepackage[T1]{fontenc}
\usepackage{fixltx2e}
\usepackage{graphicx}
\usepackage{longtable}
\usepackage{float}
\usepackage{wrapfig}
\usepackage{soul}
\usepackage{textcomp}
\usepackage{marvosym}
\usepackage{wasysym}
\usepackage{latexsym}
\usepackage{amssymb}
\usepackage{hyperref}
\tolerance=1000
\providecommand{\alert}[1]{\textbf{#1}}

\title{scales}
\author{Colin Maxwell}
\date{21 August 2013}

\begin{document}

\maketitle

\setcounter{tocdepth}{3}
\tableofcontents
\vspace*{1cm}

A column of your data is mapped onto an aesthetic. The aesthetic is
then translated into colors and positions on the plot by a `scale.'
This translation is done automatically, but you can adjust various
parts of the process by calling the `scale' functions.

For x, y, alpha, size, linetype, and aesthetics, scales come in two
flavors ``discrete'' and ``continuous.'' Continuous scales can be any real
number, whereas discrete scales can be labelled using
integers. Practically, what this means is that a discrete scale will
have every value named with a unique label, whereas continuous scales
will be labelled at periodic points with ticks.

ggplot uses a consistent naming system to make remembering all the functions
that you use to adjust these scales easy: scale_''AESTHETIC.NAME''_''SCALE.TYPE.''

So for the `x' aesthetic, you have
`scale$_x$$_{\mathrm{discrete}}$', `scale$_x$$_{\mathrm{continuous}}$', `scale$_x$$_{\mathrm{log10}}$',
`scale$_x$$_{\mathrm{reverse}}$', `scale$_x$$_{\mathrm{sqrt}}$', `scale$_x$$_{\mathrm{date}}$', etc.

\section{Choosing a discrete or continuous scale}
\label{sec-1}


Since ggplot is relatively smart, numerical columns are automatically
mapped onto continous scales. For this reason, adding on
``scale$_x$$_{\mathrm{continuous}}$'' doesn't do anything. 

\begin{verbatim}
ggplot( diamonds.sampled, aes(x=price))+
    geom_histogram()+
    scale_x_continuous()
\end{verbatim}


On the other hand, if you specified a discrete scale ggplot would
label each value in the vector individually. It would first generate a
factor of the variable `price', which would mean that there would be
(thousands) of unique variables before it tried to compute a
histogram. The results are non-sensical and are shown below. Don't run
it because it takes forever.

\begin{verbatim}
#Don't run
ggplot( diamonds.sampled, aes(x=price))+
    geom_histogram()+
    scale_x_discrete()
\end{verbatim}


Likewise, if a column contains character values, it's automatically
mapped to a discrete value:

\begin{verbatim}
ggplot( diamonds.sampled, aes(x=cut))+
    geom_histogram()+
    scale_x_discrete()
\end{verbatim}


However, the opposite will yield an error since there's not really a
good way to go from discrete to continuous values.

\begin{verbatim}
ggplot( diamonds.sampled, aes(x=cut))+
    geom_histogram()+
    scale_x_continuous()
\end{verbatim}


Everything that I mentioned about x and y scales is also true for
shape, linetype, and all the others I listed above.

\begin{verbatim}
ggplot( diamonds.sampled, aes(x=carat, y=price, shape=cut))+
    geom_point()+
    scale_shape_discrete()
\end{verbatim}
\section{Arguments to scales}
\label{sec-2}


You can specify how the scales are named by passing a string to the
appropriate scale function. This argument doesn't need to be named:

\begin{verbatim}
ggplot( diamonds.sampled, aes(x=carat, y=price, shape=cut))+
    geom_point()+
    scale_x_continuous("carrots")+
    scale_y_continuous("$$")+
    scale_shape_discrete("beauty")
\end{verbatim}


One thing that you can do with scales is determine the range of data
that you will plot. To do this, specify the limits argument in the
appropriate scale:

\begin{verbatim}
ggplot( diamonds.sampled, aes(x=carat, y=price, shape=cut))+
    geom_point()+
    scale_x_continuous(limits=c(0,1))+
    scale_y_continuous(limits=c(0,5000))
\end{verbatim}


There's also shorthand functions for this:

\begin{verbatim}
ggplot( diamonds.sampled, aes(x=carat, y=price, shape=cut))+
    geom_point()+
    xlim(0,1)+
    ylim(0,5000)
\end{verbatim}

You can also control where you want tick marks to appear:

\begin{verbatim}
ggplot( diamonds.sampled, aes(x=carat, y=price, shape=cut))+
    geom_point()+
    scale_x_continuous(breaks=(1:30)/10)
\end{verbatim}


There's also some formatting options, such as percentages, dates, or
dollars. These are found in the `scales' library and stored as
formatting functions. You tell the scale what formatting function you
want. You can also write your own. Consult the documentation for details.

\begin{verbatim}
library(scales)

ggplot( diamonds.sampled, aes(x=carat*1000, y=price, shape=cut))+
    geom_point()+
    scale_y_continuous(labels=dollar)+
    scale_x_continuous(labels=comma)
\end{verbatim}
\section{Kinds of x and y scales}
\label{sec-3}


A common task with x and y scales is to make one of the scales
logarithmic. You can do this by adding on the functions
`scale$_x$$_{\mathrm{log10}}$' or `scale$_y$$_{\mathrm{log10}}$':

\begin{verbatim}
ggplot( growth.PMY1529, aes(x=time,
                        y=OD,
                        col=strain,
                        lty=dextrose,
                        group=plate_pos))+
    geom_line()+
    scale_y_log10()+
    scale_x_log10()
\end{verbatim}


Similarly, you can do square roots automatically:

\begin{verbatim}
ggplot( growth.PMY1529, aes(x=time,
                        y=OD,
                        col=strain,
                        lty=dextrose,
                        group=plate_pos))+
    geom_line()+
    scale_y_sqrt()+
    scale_x_sqrt()
\end{verbatim}
\section{Continuous color scales}
\label{sec-4}


Color scales have a few things that are different about them that need
to be treated separately from other types of scales. The aesthetics
`fill' and `color' which control the fill and outline colors of
shapes, respectively, also have two different types of scales
depending on whether the column of data they map is discrete or
continuous. However, you can make discrete and continuous color scales
in several differnt ways, which leads to several different functions.

Continuous color mappings default to mapping between blue and black:

\begin{verbatim}
ggplot(pp(100), aes(x=x,y=y, fill=z))+geom_tile()+
    scale_fill_gradient()
\end{verbatim}


You can change the two colors that mark the high and low end of the
data by specifying the arguments `high' and `low' to the function
scale$_{\mathrm{fill}}$$_{\mathrm{gradient}}$ (or scale$_{\mathrm{color}}$$_{\mathrm{gradient}}$ depending on the aesthetic)

\begin{verbatim}
ggplot(pp(100), aes(x=x,y=y, fill=z))+geom_tile()+
    scale_fill_gradient(low = "blue", high = "yellow")
\end{verbatim}


If you want three colors, use `scale$_{\mathrm{fill}}$$_{\mathrm{gradient2}}$'

\begin{verbatim}
ggplot(pp(100), aes(x=x,y=y, fill=z))+geom_tile()+
    scale_fill_gradient2(low = "blue", mid = "black", high = "yellow")
\end{verbatim}


For more use `scale$_{\mathrm{fill}}$$_{\mathrm{gradientn}}$'

\begin{verbatim}

#Annoyingly, you have to use the British spelling of 'color'

ggplot(pp(100), aes(x=x,y=y, fill=z))+geom_tile()+
    scale_fill_gradientn(colours = c("blue", "black", "yellow", "red"))
\end{verbatim}
\section{Discrete colors}
\label{sec-5}


Discrete color scales are just like discrete x or y values. If the
column is a character vector, the default scale will be discrete.

The default discrete ggplot colors are some easy to distinguish pastels.

\begin{verbatim}
ggplot( growth.270, aes(x=time,
                        y=OD,
                        col=strain,
                        group=plate_pos))+
    geom_line()
\end{verbatim}


You can easily change this to greyscale with `scale$_{\mathrm{color}}$$_{\mathrm{grey}}$':

\begin{verbatim}
ggplot( growth.270, aes(x=time,
                        y=OD,
                        col=strain,
                        group=plate_pos))+
    geom_line()+
    scale_color_grey()
\end{verbatim}


Just like with axis labels, this is where you can change what the
scale is called:

\begin{verbatim}
ggplot( growth.270, aes(x=time,
                        y=OD,
                        col=strain,
                        group=plate_pos))+
    geom_line()+
    scale_color_grey("Magwene lab\nstrain number")
\end{verbatim}



If you don't like the default R colors, but don't want to bother to
specify your own, the package RColorBrewer has some nice defaults. The
following code displays the available color scales:

You can use the function ``display.brewer.all'' to plot all the scales
that are available for plotting.

\begin{verbatim}
library(RColorBrewer)
display.brewer.all()
\end{verbatim}


ggplot lets you choose what palette you want to use by either an
integer and the type of scale (sequential, qualitative, or diverging;
the groups above correspond to these types):

\begin{verbatim}
ggplot( growth.270, aes(x=time,
                        y=OD,
                        col=strain,
                        group=plate_pos))+
    geom_line()+
    scale_color_brewer(palette = 2, type = "seq")
\end{verbatim}


\begin{verbatim}
ggplot( growth.270, aes(x=time,
                        y=OD,
                        col=strain,
                        group=plate_pos))+
    geom_line()+
    scale_color_brewer(palette = 2, type = "div")
\end{verbatim}


\begin{verbatim}
ggplot( growth.270, aes(x=time,
                        y=OD,
                        col=strain,
                        group=plate_pos))+
    geom_line()+
    scale_color_brewer(palette = 2, type = "qual")
\end{verbatim}



Or the name of the palette (shown above):

\begin{verbatim}
ggplot( growth.270, aes(x=time,
                        y=OD,
                        col=strain,
                        group=plate_pos))+
    geom_line()+
    scale_color_brewer(palette = "Set1")
\end{verbatim}

\end{document}