% Created 2013-08-21 Wed 17:11
\documentclass[11pt]{article}
\usepackage[utf8]{inputenc}
\usepackage[T1]{fontenc}
\usepackage{fixltx2e}
\usepackage{graphicx}
\usepackage{longtable}
\usepackage{float}
\usepackage{wrapfig}
\usepackage{soul}
\usepackage{textcomp}
\usepackage{marvosym}
\usepackage{wasysym}
\usepackage{latexsym}
\usepackage{amssymb}
\usepackage{hyperref}
\tolerance=1000
\providecommand{\alert}[1]{\textbf{#1}}

\title{facets}
\author{Colin Maxwell}
\date{21 August 2013}

\begin{document}

\maketitle

\setcounter{tocdepth}{3}
\tableofcontents
\vspace*{1cm}

One of the coolest things about ggplots is the ability to split the
data into facets before making a plot. This is a really good way to
convey data because you only have to remember one set of axes but you
get to see lots of subsets of the data. For example, to reduce
overplotting in one of the plots we made above, we could split our data by the type of media:

\begin{verbatim}
ggplot( growth.270, aes(x=time,
                        y=OD,
                        col=strain,
                        group=plate_pos))+
    geom_line()+
    facet_wrap(~media)
\end{verbatim}


This plot very naturally compares between the different strains. We
can easily reverse it, though and compare between medias:

\begin{verbatim}
ggplot( growth.270, aes(x=time,
                        y=OD,
                        col=media,
                        group=plate_pos))+
    geom_line()+
    facet_wrap(~strain)
\end{verbatim}


There's two kinds of faceting that you can do: ``facet$_{\mathrm{wrap}}$'' and
``facet$_{\mathrm{grid}}$''. Facet wrap makes a ribbon of tiles in order to split
your data. This is most appropriate when the facets don't need to be
plotted in any particular order. On the other hand, facet$_{\mathrm{grid}}$ can be
used if the order does matter. For example we can make a grid of
ammonium and dextrose concentrations:

\begin{verbatim}
ggplot( growth.270, aes(x=time,
                        y=OD,
                        col=strain,
                        group=plate_pos))+
    geom_line()+
    facet_grid(dextrose~ammonium, as.table=TRUE)
\end{verbatim}


Notice that the facets are a bit out of order. I wanted the `high
ammonium, high dextrose' condition in the upper left corner. The order
that facets go in (and in general, the order that discrete scales are
plotted in) is determined by the order of their levels. To get
the behavior we want, we just have to specify the order that we want
the levels to go in:

\begin{verbatim}
growth.270 <- subset(growth,(initial_dilution == 270))
growth.270$ammonium <- factor(growth.270$ammonium, levels = c("L ammonium", "H ammonium"))
growth.270$dextrose <- factor(growth.270$dextrose, levels = c("H dextrose", "L dextrose"))


ggplot( growth.270, aes(x=time,
                        y=OD,
                        col=strain,
                        group=plate_pos))+
    geom_line()+
    facet_grid(dextrose~ammonium)
\end{verbatim}


Another neat option is to get an additional column corresponding to
the margins of the faceted variables:

\begin{verbatim}
ggplot( growth.270, aes(x=time,
                        y=OD,
                        col=strain,
                        group=plate_pos))+
    geom_line()+
    facet_grid(dextrose~ammonium, margins = TRUE)
\end{verbatim}


In this case, adding the margins makes clear that changing the
dextrose concentrations in the presence of low or high ammonium doesn't change
much in terms of growth rate but that changing the ammonium
concentrations in the presence of low or high dextrose leads to higher
growth rate as well as a higher saturation density.

You can also add additional variables onto the facetting variables in
order to display additional facets of the data. For example, here is
the original plate map for the growth data:


\begin{verbatim}
ggplot( growth, aes(x=time,y=OD,label=strain, group=plate_pos))+geom_line()+
    facet_grid(initial_dilution+row~col+ammonium+dextrose+strain)
\end{verbatim}


\end{document}