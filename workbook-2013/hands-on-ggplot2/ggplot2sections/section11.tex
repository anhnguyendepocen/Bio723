% Created 2013-08-21 Wed 17:12
\documentclass[11pt]{article}
\usepackage[utf8]{inputenc}
\usepackage[T1]{fontenc}
\usepackage{fixltx2e}
\usepackage{graphicx}
\usepackage{longtable}
\usepackage{float}
\usepackage{wrapfig}
\usepackage{soul}
\usepackage{textcomp}
\usepackage{marvosym}
\usepackage{wasysym}
\usepackage{latexsym}
\usepackage{amssymb}
\usepackage{hyperref}
\tolerance=1000
\providecommand{\alert}[1]{\textbf{#1}}

\title{How to define your own ggfunction?}
\author{Colin Maxwell}
\date{21 August 2013}

\begin{document}

\maketitle

\setcounter{tocdepth}{3}
\tableofcontents
\vspace*{1cm}

But what if you wanted to make a whole series of these plots for each
of the combinations of the variables? In this case, you would want to
write a function. You might think that you could just replace the
variable `iris' with `plotData' and go from there. However, this
approach yields an error:

\begin{verbatim}

scatterWithMargins <- function(plotData, x, y, color){
    xy <- ggplot(plotData, aes(x=x, y=y, col = color))+
        geom_point()+
        geom_density2d()+
        theme_classic()
    xDen <- ggplot(plotData, aes(x=x, fill = color, col=color))+
        geom_density(alpha=0.1)+
        theme_classic()+
        xlab("")
    yDen <- ggplot(plotData, aes(x=y, fill = color, col=color))+
        geom_density(alpha=0.1)+
        coord_flip()+
        theme_classic()+
        xlab("")
    leg <- g_legend( xDen )
    grid.arrange(xDen+theme(legend.position = "none"),
                 leg,
                 xy+theme(legend.position = "none"),
                 yDen+theme(legend.position = "none"),
                 nrow = 2, ncol = 2,
                 widths = unit(c(2,1), c("null", "null")),
                 heights = unit(c(1,2), c("null", "null"))
                 )
}

scatterWithMargins(iris, Sepal.Length, Sepal.Width, Species)
\end{verbatim}


This is because when the function ggplot sets itself up, it always
looks in the global environment for variables. That means that even
though you told it to look in the data frame `plotData', it's actually
trying to find the variable `x' in the global environment. Again, it
doesn't matter if you understand R well enough for that to mean much
to you, the upshot is that when you use ggplot in a function, you need
to specify a different kind of `aes' in the ggplot function call:
`aes$_{\mathrm{string}}$'. aes$_{\mathrm{string}}$ sets up aesthetic mappings with strings that
name the columns of the data that you want the aesthetic set to. This
means that the variable names must always be quoted. When you make
these changes, your function will work hunky dory.

\begin{verbatim}

scatterWithMargins <- function(plotData, x, y, color){
    xy <- ggplot(plotData, aes_string(x=x, y=y, col = color))+
        geom_point()+
        geom_density2d()+
        theme_classic()
    xDen <- ggplot(plotData, aes_string(x=x, fill = color, col=color))+
        geom_density(alpha=0.1)+
        theme_classic()+
        xlab("")
    yDen <- ggplot(plotData, aes_string(x=y, fill = color, col=color))+
        geom_density(alpha=0.1)+
        coord_flip()+
        theme_classic()+
        xlab("")
    leg <- g_legend( xDen )
    grid.arrange(xDen+theme(legend.position = "none"),
                 leg,
                 xy+theme(legend.position = "none"),
                 yDen+theme(legend.position = "none"),
                 nrow = 2, ncol = 2,
                 widths = unit(c(2,1), c("null", "null")),
                 heights = unit(c(1,2), c("null", "null"))
                 )
}

scatterWithMargins(iris, "Sepal.Length", "Sepal.Width", "Species")
\end{verbatim}


\end{document}