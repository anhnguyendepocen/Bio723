% Created 2013-08-21 Wed 17:09
\documentclass[11pt]{article}
\usepackage[utf8]{inputenc}
\usepackage[T1]{fontenc}
\usepackage{fixltx2e}
\usepackage{graphicx}
\usepackage{longtable}
\usepackage{float}
\usepackage{wrapfig}
\usepackage{soul}
\usepackage{textcomp}
\usepackage{marvosym}
\usepackage{wasysym}
\usepackage{latexsym}
\usepackage{amssymb}
\usepackage{hyperref}
\tolerance=1000
\providecommand{\alert}[1]{\textbf{#1}}

\title{Using ggplot2 to make a scatter plot}
\author{Colin Maxwell}
\date{21 August 2013}

\begin{document}

\maketitle

\setcounter{tocdepth}{3}
\tableofcontents
\vspace*{1cm}
The goal of this section is to motivate why you might want to use
ggplot rather than the R base graphics. To begin with, we'll load the
package ggplot2 that contains the functions we'll need for the first
section and two data.frames of data: we've already seen the iris
dataset. The second dataset is the color, cut, clarity, caret and
price of about 60,000 diamonds. Since the diamonds data is so big,
we'll take a subsample of 1,000 points:

\begin{verbatim}
library(ggplot2); data(iris); data(diamonds)
diamonds.sampled <- diamonds[ sample(1:nrow(diamonds), 1000),]
\end{verbatim}

As a first example, let's think about a scatter plot, to begin with,
  we'll use the R base package. When we tell R to make a scatterplot,
  we have to tell it what the `X' values will be and what the `Y'
  values will be:

\begin{verbatim}
plot( x = iris$Sepal.Length, y = iris$Petal.Length)
\end{verbatim}


You can imagine an algorithm that tells R to go to each line in the
  table of values, find the column for the Sepal length, then the
  column for the petal length, and finally, to draw a dot where those
  two things intersect. Another way of saying this is that the
  horizonal coordinates are mapped onto a particular column of the
  data (the sepal length) and the vertical coordinates are mapped onto
  another column (the petal length).

 We could embellish our plot a bit by adding some colors to
  correspond to the species. However, this actually involves a fair
  bit of somewhat obscure code:


\begin{verbatim}
#Set up a vector of color keywords for R to index
COLORS = c("red", "blue", "green")

#Since factors are encoded as numbers and a list of levels, we can index
#the color vector using the numeric version of the species
plot( x = iris$Sepal.Length, y = iris$Petal.Length,
      col = COLORS[ as.numeric( iris$Species ) ] )

#Finally, we can add a legend
legend( "topleft", pch = 1, col = COLORS, legend = levels( iris$Species ) )
\end{verbatim}


Now we have three mappings: one from sepal length to X, one from
  petal length to Y, and one from species to color. In the lingo of
  ggplots, these mappings are called `aesthetic mappings' and the way
  that these mappings are drawn on the plot is called the `geometry,'
  or the `geom.' Using ggplot, we can make the same plot, but with a
  little different syntax:

\begin{verbatim}
ggplot(iris, aes(x = Sepal.Length,
                 y = Petal.Length,
                 col = Species)
       )+
    geom_point()
\end{verbatim}


The function `ggplot' takes two arguments: the first is a data.frame
  that your plot will be made out of (iris in this case). The second is a function called
  `aes' that sets up the aesthetic mappings. In this case, we told
  ggplot that we wanted the sepal length on the x axis, the petal
  length on the y axis, and the colors to be encoded by the
  species. However, we could choose any number of other aesthetic
  mappings. We could use shape:

\begin{verbatim}
ggplot(iris, aes(x = Sepal.Length,
                 y = Petal.Length,
                 shape = Species)
       )+
    geom_point()
\end{verbatim}



Or size:

\begin{verbatim}
ggplot(iris, aes(x = Sepal.Length,
                 y = Petal.Length,
                 size = Species)
       )+
    geom_point()
\end{verbatim}


ggplot is also smart enough to render continuous variables as colors or sizes:

\begin{verbatim}
ggplot(iris, aes(x = Sepal.Length,
                 y = Petal.Length,
                 size = Petal.Width)
       )+
    geom_point()
\end{verbatim}


\begin{verbatim}
ggplot(iris, aes(x = Sepal.Length,
                 y = Petal.Length,
                 col = Petal.Width)
       )+
    geom_point()
\end{verbatim}


We could even make a difficult to interpret plot showing all five
  data columns:

\begin{verbatim}
ggplot(iris, aes(x = Sepal.Length,
                 y = Petal.Length,
                 col = Petal.Width,
                 size = Sepal.Width,
                 shape = Species)
       )+
    geom_point()
\end{verbatim}


There's quite a lot of advantages to this approach rather than trying
to replicate this plot with base graphics:

\begin{enumerate}
\item the legend is automatically drawn for you.
\item the code is very easy to change. Rather than having to figure out
   how to manually map a point size onto a variable using some
   difficult R code, it's just as simple as saying to set the `size'
   equal to a `variable'.
\item it's easy to swap around variables from one aesthetic mapping to another.
\end{enumerate}

And this is just the beginning!



\end{document}