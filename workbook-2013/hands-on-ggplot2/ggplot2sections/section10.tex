% Created 2013-08-21 Wed 17:11
\documentclass[11pt]{article}
\usepackage[utf8]{inputenc}
\usepackage[T1]{fontenc}
\usepackage{fixltx2e}
\usepackage{graphicx}
\usepackage{longtable}
\usepackage{float}
\usepackage{wrapfig}
\usepackage{soul}
\usepackage{textcomp}
\usepackage{marvosym}
\usepackage{wasysym}
\usepackage{latexsym}
\usepackage{amssymb}
\usepackage{hyperref}
\tolerance=1000
\providecommand{\alert}[1]{\textbf{#1}}

\title{multiple plots per page}
\author{Colin Maxwell}
\date{21 August 2013}

\begin{document}

\maketitle

\setcounter{tocdepth}{3}
\tableofcontents
\vspace*{1cm}

If you are using base graphics, you can set the number of plots per
page using the `mfrow' option to the function `par':

\begin{verbatim}
par(mfrow = c(2,2))
plot( x = iris$Sepal.Length, y = iris$Petal.Length, col = COLORS[ as.numeric( iris$Species ) ])
plot( x = iris$Sepal.Width, y = iris$Petal.Length, col = COLORS[ as.numeric( iris$Species ) ])
plot( x = iris$Sepal.Length, y = iris$Petal.Width, col = COLORS[ as.numeric( iris$Species ) ])
plot( x = iris$Sepal.Width, y = iris$Petal.Width, col = COLORS[ as.numeric( iris$Species ) ])
\end{verbatim}


However, if you try this with ggplot, you'll end up with one plot per
page. This is because ggplot makes use of a different plotting system
than base graphics called `grid.' There's a couple ways to get
multiple plots in a page. You read the `grid' documentation and figure
out how to use the `viewports' functions. Or, there's a convenience
function called `grid.arrange' in the `gridExtra' package that makes
this task much simpler.

One nice thing about ggplots is that it lets you save your plots as
variables without evaluation for plotting later. This makes code much
more readable in some situations. To begin with, we'll define the four
plots we made above with ggplot:

\begin{verbatim}
p1 <- ggplot(iris, aes(x=Sepal.Length, y = Petal.Length, col=Species))+geom_point()
p2 <- ggplot(iris, aes(x=Sepal.Width, y = Petal.Length, col=Species))+geom_point()
p3 <- ggplot(iris, aes(x=Sepal.Width, y = Petal.Width, col=Species))+geom_point()
p4 <- ggplot(iris, aes(x=Sepal.Length, y = Petal.Width, col=Species))+geom_point()
\end{verbatim}


Each of these plots is now stored in their respective variables. You
can plot them using the `print' function:

\begin{verbatim}
print(p1)
\end{verbatim}


Now that we have four plots stored, we can print them in rows using
the `grid.arrange' function:

\begin{verbatim}
library(grid); library(gridExtra)

grid.arrange(p1, p2, p3, p4,
             nrow=2)
\end{verbatim}


This is pretty nice, but it would be good to just plot the legend
once. To do this, we'll first suppress the legends for the four plots
we made using theme:

\begin{verbatim}
p1 <- p1+theme(legend.position = "none")
p2 <- p2+theme(legend.position = "none")
p3 <- p3+theme(legend.position = "none")
p4 <- p4+theme(legend.position = "none")
\end{verbatim}

Next, we'll use a function to pull out only the legend from the
function call that made the orignal p1.

\begin{verbatim}
g_legend<-function(a.gplot){
    tmp <- ggplot_gtable(ggplot_build(a.gplot))
    leg <- which(sapply(tmp$grobs, function(x) x$name) == "guide-box")
    legend <- tmp$grobs[[leg]]
    return(legend)
}
p1.original = ggplot(iris, aes(x=Sepal.Length, y = Petal.Length, col=Species))+geom_point()
plotLegend = g_legend(p1.original)
\end{verbatim}

The next code is a little confusing, but what it does is simple. In
order to compute how wide the legend column in the plot should be, we
sum the attribute `width' of the object plot legend. Next we combine
the four plots we defined above using the function `arrangeGrob.' The
function works just like grid.arrange, except you get to save the
object. This results in a combined plot of the four plots in a grid
saved as `combinedPlots.'

Finally, we call grid.arrange with the two plots we want: the grid of
four plots and the legend. The only tricky part is that we need to
tell grid how wide to make everything. The function unit.c combines
unit objects analogously to the way that the function `c' combines
numbers. The first argument to unit.c is ``unit(1, ``npc'') - lwidth''. ``npc'' stands
for normalized parental coordinates, which just means the size of the
plot. Since the plot is size 1 in npc, subtracting the plot width
gives us how much room we need for our combined plots. Likewise the
second argument is how wide we want the legend to be, which is just
`lwidth'.

\begin{verbatim}
lwidth = sum(plotLegend$width)
combinedPlots <- arrangeGrob(p1, p2, p3, p4, nrow=2)

grid.arrange(combinedPlots,
             plotLegend, 
             widths=unit.c(unit(1, "npc") - lwidth, lwidth), nrow=1
             )
\end{verbatim}


So that's all very complicated, why go over it? Basically so that I can
show you how to make my favorite ggplot: a scatterplot with marginal
density plots. First, we'll define three plots and a legend. Note that
the yDen plot has had it's coordinates flipped (the x becomes y and
vice versa) using the function coord$_{\mathrm{flip}}$. This is because it's going
to show the marginal distribution of y values and we want the density
to be oriented `up-down' and not `left-right.' Also notice that I
tacked on geom$_{\mathrm{density2d}}$ to the xy plot. You'll see what that does below.

\begin{verbatim}
xy <- ggplot(iris, aes(x=Petal.Width, y=Sepal.Length, col = Species))+
    geom_point()+
    geom_density2d()+
    theme_classic()
xDen <- ggplot(iris, aes(x=Petal.Width, fill = Species, col=Species))+
    geom_density(alpha=0.1)+
    theme_classic()+
    xlab("")
yDen <- ggplot(iris, aes(x=Sepal.Length, fill = Species, col=Species))+
    geom_density(alpha=0.1)+
    coord_flip()+
    theme_classic()+
    xlab("")
leg <- g_legend( xDen )
\end{verbatim}

Next, we'll use grid.arrange to put everything in the order we want.

\begin{verbatim}
grid.arrange(xDen+theme(legend.position = "none"),
             leg,
             xy+theme(legend.position = "none"),
             yDen+theme(legend.position = "none"),
             nrow = 2, ncol = 2,
             widths = unit(c(2,1), c("null", "null")),
             heights = unit(c(1,2), c("null", "null"))
             )
\end{verbatim}


The result is pretty cool! You can see how the data correlate with
each other as well as how well each variable differentiates between
the different groups.

\end{document}