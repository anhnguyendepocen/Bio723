

\section{Writing Functions in Python}

The general form of a Python function is as follows:
%
\begin{python}
def funcname(arg1,arg2):
    # one or more expressions
    return someresult # arbitrary python object (could even be another function)
\end{python}
%
An important thing to remember when writing functions is that Python is
white space sensitive. In Python code indentation indicates scoping
rather than braces. Therefore you need to maintain consistent
indentation. This may surprise those of you who have extensive
programming experience in another language. However, white space
sensitivity contributes significantly to the readability of Python code.
Use a Python aware programmer's editor and it will become second nature
to you after a short while. I recommend you set your editor to
substitute spaces for tabs (4 spaces per tab), as this is the default
convention within the python community.

Here's an example of defining and using a function in the Python
interpreter:
%
\begin{python}
>>> def mypyfunc(x,y):
...     return x**2 + y**2 + 3*x*y
... 
>>> mypyfunc(10,12)
604
>>> a = numpy.arange(1,5,0.5)
>>> b = numpy.arange(2,6,0.5)
>>> mypyfunc(a,b)
array([  11.  ,   19.75,   31.  ,   44.75,   61.  ,   79.75,  101.  ,
        124.75])
>>> a = range(1,5)
>>> b = range(1,5)
>>> mypyfunc(a,b)

Traceback (most recent call last):
  File "<pyshell#52>", line 1, in -toplevel-
    mypyfunc(a,b)
  File "<pyshell#45>", line 2, in mypyfunc
    return x**2 + y**2 + 3*x*y
TypeError: unsupported operand type(s) for ** or pow(): 'list' and 'int'
>>> 
\end{python}
%
Note that this function works for numeric types (\lstinline!ints! and
\lstinline!floats!) as well as \lstinline!numpy.arrays! but not for
simple Python lists. If you wanted to make this function work for lists
as well you could define the function as follows:
%
\begin{python}
>>> def mypyfunc(x,y):
...     x = numpy.array(x)
...     y = numpy.array(y)
...     return x**2 + y**2 + 3*x*y
... 
>>> a
[1, 2, 3, 4]
>>> b
[1, 2, 3, 4]
>>> mypyfunc(a,b)
array([ 5, 20, 45, 80])
\end{python}

\subsection{Putting Python functions in Modules}

As with R, you can define your own Python modules that contain user
defined functions. Using a programmer's text editor, write your
function(s) and save it to a file with a \lstinline!.py! extension in a
directory in your PYTHONPATH (see above).

\begin{python}
# functions defined in vecgeom.py
import numpy

def veclength(x):
    """Calculate length of a vector x.""" 
    x = numpy.array(x)
    return numpy.sqrt(numpy.dot(x,x))


def unitvector(x):
    """Return a unit vector in the same direction as x."""
    x = numpy.array(x)
    return x/veclength(x)
\end{python}
%
To access your function use an \lstinline!import! statement:
%
\begin{python}
>>> import vecgeom
>>> x = [-3,-3,-1,-1,0,0,1,2,2,3]
>>> help(vecgeom.veclength)
Help on function veclength in module vecgeom:

veclength(x)
    Calculate length of a vector x.

>>> vecgeom.veclength(x)
6.164414002968976
# import all fxns from the vecgeom module
>>> from vecgeom import * 
>>> print vecgeom.unitvector(x)
[-0.48666426 -0.48666426 -0.16222142 -0.16222142  0.          0.   
     0.16222142  0.32444284  0.32444284  0.48666426]
\end{python}