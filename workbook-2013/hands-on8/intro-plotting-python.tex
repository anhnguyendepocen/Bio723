
\section{Plotting in Python}

Python doesn't have any `native' data plotting tools but there are a
variety of packages that provide tools for visualizing data. The Matplotlib package is the de facto standard for producing
publication quality scientific graphics in Python. Matplotlib is
included with the EPD and was automatically pulled into the interpretter
namespace if you're using the \ipython |--pylab| option. If you want to explore the full power of Matplotlib check out the example
gallery and the documentation at
\url{http://matplotlib.sourceforge.net/}.

\subsection{Basic plots using matplotib}

If you invoked the Ipython shell using the pylab option than most of the
basic matplotlib functions are already available to you. If not, import
them as so:

\begin{python}
>>> from pylab import * # only necessary if not using pylab
>>> import numpy as np # go ahead and import numpy as well, using an alias

# load the turtle data using the numpy.loadtxt function 
# skipping the first row (header) and the first column 
>>> turt = np.loadtxt('turtles.txt', skiprows=1, 
                      usecols=(1,2,3))
>>> turt.shape
(48, 3)
# draw bivariate scatter plot
>>> scatter(turt[:,0], turt[:,1])
# give the axes some labels and a title for the plot
>>> xlabel('Length')
>>> ylabel('Width')
>>> title('Turtle morphometry')
\end{python}

Here's another example using the yeast expression data set:
%
\begin{python}
>>> data = np.loadtxt('yeast-subnetwork-clean.txt',skiprows=1,usecols=range(1,16))
>>> data.shape   # check the dimensions of the resulting matrix
(173, 15)
\end{python}
The \lstinline!skiprows! argument tells the function how many rows in
the data file you want to skip. In this case we skipped only the first
row which gives the variable names. The \lstinline!usecols! arguments
specificies which columns from the data file to use. Here we skipped the
first (zeroth) column which had the names of the conditions. The usecols
\lstinline!loadtxt! works when there is no missing data. Use
\lstinline!numpy.genfromtxt! instead when there are missing values. For
a full tutorial on how to use the \lstinline!numpy.genfromtxt! function
see
\url{http://docs.scipy.org/doc/numpy/user/basics.io.genfromtxt.html}.

\subsubsection{Histograms in Matplotlib}

Matplotlib has a histogram drawing function. Here's how to use it:
%
\begin{python}
>>> hist? # in Ipython calls the help function
>>> h = hist(data[:,0]) # plot a histogram of the first variable (column) in our data set
>>> clf() # clear the plot window, don't need this if you closed the plot window
>>> h = hist(data[:,0], bins=20) # plot histogram w/20 bins
>>> h = hist(data[:,:2])  # histograms of the first two variables    
\end{python}

There's no built in density plot function, but we can create a function
that will do the necessary calculations for us to create our own density
plot. This uses a kernel density estimator function in the scipy library
(included with EPD). Put the following code in a file called
\lstinline!myplots.py! somewhere on your \lstinline!PYTHONPATH!:
%
\begin{python}
# myplots.py

import numpy as np
from scipy import stats

def density_trace(x):
    kde = stats.gaussian_kde(x)
    xmin,xmax = min(x), max(x)
    xspan = xmax - xmin
    xpts = np.arange(xmin, xmax, xspan/1000.)
    ypts = kde.evaluate(xpts) # evalude the estimate at the xpts
    return xpts,ypts
\end{python}

You can then use the \lstinline!density_trace! function as follows:
%
\begin{python}
>>> import myplots
>>> h = hist(data[:,0], normed=True) # use normed=True so histogram 
                           # is normalized to form a prob. density
>>> x,y = myplots.density_trace(data[:,0])
>>> plot(x,y, 'red')    
\end{python}

\subsubsection{Boxplots in Matplotlib}

Box-and-whisker plots are straightforward in Matplotlib:
%
\begin{python}
>>> b = boxplot(data[:,0])
>>> clf()
>>> b = boxplot(data[:,:5]) # boxplots of first 5 variables
\end{python}
%
The \lstinline!boxplot! function has quite a few facilities for
customizing your boxplots. For example, here's how we can create a
notched box-plot using 1000 bootstrap replicates (we'll discuss the
bootstrap in more detail in a later lecture) to calculate confidence
intervals for the median.
%
\begin{python}
>>> boxplot(data[:,0], notch=1, bootstrap=True)    
\end{python}
See the Matplotib docs for more info.

\subsubsection{3D Scatter Plots in Matplotlib}

Recent version of Matplotlib include facilities for creating 3D plots.
Here's an example of a 3D scatter plot:

\begin{python}
>>> from mpl_toolkits.mplot3d import Axes3D
>>> fig = figure()
>>> ax = fig.add_subplot(111, projection = '3d')
>>> ax.scatter(data[:,0],data[:,1],data[:,2])
<mpl_toolkits.mplot3d.art3d.Patch3DCollection object at 0x1a0bbd70>
>>> ax.set_xlabel('Gene 1')
<matplotlib.text.Text object at 0x1a0ae7d0>
>>> ax.set_ylabel('Gene 2')
<matplotlib.text.Text object at 0x1a0bb2b0>
>>> ax.set_zlabel('Gene 3')
<matplotlib.text.Text object at 0x1a0bbcd0>
>>> show()
\end{python}
%
Retyping all those commands is tedious and error prone so let's turn it
into a function. Add the following code to \lstinline!myplots.py!:

\begin{python}
from matplotlib import pyplot
from mpl_toolkits.mplot3d import Axes3D

def scatter3d(x,y,z, labels=None):
    fig = pyplot.figure()
    ax = fig.add_subplot(111, projection='3d')
    ax.scatter(x,y,z)

    if labels is not None:
        try:
            ax.set_xlabel(labels[0])
            ax.set_ylabel(labels[1])
            ax.set_zlabel(labels[2])
        except IndexError:
            print "You specificied less than 3 labels."
    return fig
\end{python}
Now reload myplots and call the scatter3d function as so:

\begin{python}
>>> reload(myplots)
>>> myplots.scatter3d(data[:,0], data[:,1], data[:,2])
>>> myplots.scatter3d(data[:,0], data[:,1], data[:,2], lab)
>>> myplots.scatter3d(data[:,0], data[:,1], data[:,2],labels=('X','Y','Z'))
\end{python}
