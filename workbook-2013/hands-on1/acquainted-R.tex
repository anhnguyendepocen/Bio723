\section{Getting Acquainted with R}

\subsection{Installing R}


The R website is at \url{http://www.r-project.org/}. I recommend that you spend a few minutes checking out the resources, documentation, and links on this page.   Download the appropriate R installer for your computer from the Comprehensive R Archive Network (CRAN). A direct link can be found at: \url{http://cran.stat.ucla.edu/}. As of mid August 2013 the latest R release is verison 3.0.1.

The R installer will install appropriate icons under the |Start Menu| (Windows) or |Applications| Folder (OS X). On OS X it will install two icons -- ``R" and ``R64", corresponding to 32-bit and 64-bit versions of the executable.  The 64 bit version, which allows access to much larger amounts of your comptuer's RAM, is suitable for dealing with very large data sets. 


\subsection{Starting and R Interactive Session}

 The \OSX and Windows version of R provide a simple GUI
interface for using R in interactive mode. When you start up the R GUI you'll be
presented with a single window, the R console. See the your textbook, The Art of R Programming (AoRP) for a discussion of the difference between R's interactive and batch modes.

\subsection{Accessing the Help System on R}

R comes with fairly extensive documentation and a simple help system. You can
access HTML versions of R documentation under the Help menu in the GUI. The
HTML documentation also includes information on any packages you've installed.
Take a few minutes to browse through the R HTML documentation.

The help system can be invoked from the console itself using the
|help| function or the |?| operator.
%
\begin{R}
> help(length)
> ?length
> ?log
\end{R}
%
What if you don't know the name of the function you want? You can use
the |help.search()| function.
%
\begin{R}
> help.search("log")
\end{R}
%
In this case |help.search("log")| returns all the functions with
the string `log' in them. For more on |help.search| type
|?help.search|. Other useful help related functions include
|apropos()| and |example()|.

\subsection{Navigating Directories in R}

When you start the R environment your `working directory' (i.e.~the
directory on your computer's file system that R currently `sees')
defaults to a specific directory. On Windows this is usually the same
directory that R is installed in, on \OSX it is typically your home
directory. Here are examples showing how you can get information about
your working directory and change your working directory.
%
\begin{R}
> getwd()
[1] "/Users/pmagwene"
> setwd("/Users")
> getwd()
[1] "/Users"
\end{R}
%
Note that on Windows you can change your working directory by using the
|Change dir...| item under the |File| menu, while the corresponding item is found under the |Misc| menu on \OSX.

To get a list of the file in your current working directory use the
|list.files()| function.
%
\begin{R}
> list.files()
[1] "Shared" "pmagwene"
\end{R}



\subsection{Using R as a Calculator}

The simplest way to use R is as a fancy calculator.
%
\begin{R}
> 10 + 2 # addition
[1] 12
> 10 - 2 # subtraction
[1] 8
> 10 * 2 # multiplication
[1] 20
> 10 / 2 # division
[1] 5
> 10 ^ 2 # exponentiation
[1] 100
> 10 ** 2 # alternate exponentiation
[1] 100
> sqrt(10) # square root
[1] 3.162278
> 10 ^ 0.5 # same as square root
[1] 3.162278
> exp(1) # exponential function
[1] 2.718282
> 3.14 * 2.5^2
[1] 19.625
> pi * 2.5^2 # R knows about some constants such as Pi
[1] 19.63495
> cos(pi/3)
[1] 0.5
> sin(pi/3)
[1] 0.8660254
> log(10)
[1] 2.302585
> log10(10) # log base 10
[1] 1
> log2(10) # log base 2
[1] 3.321928
> (10 + 2)/(4-5)
[1] -12
> (10 + 2)/4-5 # compare the answer to the above
[1] -2
\end{R}
%
Be aware that certain operators have precedence over others. For example
multiplication and division have higher precedence than addition and
subtraction. Use parentheses to disambiguate potentially confusing
statements.

\begin{R}
> sqrt(pi)
[1] 1.772454
> sqrt(-1)
[1] NaN
Warning message:
NaNs produced in: sqrt(-1)
> sqrt(-1+0i)
[1] 0+1i
\end{R}
%
What happened when you tried to calculate |sqrt(-1)|?, -1 is
treated as a real number and since square roots are undefined for the
negative reals, R produced a warning message and returned a special
value called |NaN| (Not a Number). Note that square roots of
negative complex numbers are well defined so |sqrt(-1+0i)|
works fine.
%
\begin{R}
> 1/0
[1] Inf
\end{R}
Division by zero produces an object that represents infinite numbers.

\subsection{Comparison Operators}

You've already been introduced to the most commonly used arithmetic
operators. Also useful are the comparison operators:
%
\begin{R}
> 10 < 9  # less than
[1] FALSE
> 10 > 9  # greater than
[1] TRUE
> 10 <= (5 * 2) # less than or equal to
[1] TRUE
> 10 >= pi # greater than or equal to
[1] TRUE
> 10 == 10 # equals
[1] TRUE
> 10 != 10 # does not equal
[1] FALSE
> 10 == (sqrt(10)^2) # Surprised by the result? See below.
[1] FALSE
> 4 == (sqrt(4)^2) # Even more confused?
[1] TRUE
\end{R}
%
Comparisons return boolean values. Be careful to distinguish between
|==| (tests equality) and |=| (the alternative
assignment operator equivalent to |<-|).

How about the last two statement comparing two values to the square of
their square roots? Mathematically we know that both
$(\sqrt{10})^2 = 10$ and $(\sqrt{4})^2 = 4$ are true statements. Why
does R tell us the first statement is false? What we're running into
here are the limits of computer precision. A computer can't represent
$\sqrt 10$ exactly, whereas $\sqrt 4$ can be exactly represented.
Precision in numerical computing is a complex subject and beyond the
scope of this course. Later in the course we'll discuss some ways of
implementing sanity checks to avoid situations like that illustrated
above.

\subsection{Working with Vectors in R}

Vectors are the core data structure in R. Vectors store an ordered list of items all of the same type. Learning to compute effectively with vectors and one of the keys to efficient R programming.  Vectors in R always have a length (accessed with the |length()| function) and a type (accessed with the |typeof()| function).  

The simplest way to create a vector at the interactive prompt is to use the |c()| function, which is short hand for `combine' or `concatenate'.

\begin{R}
> x <- c(2,4,6,8)
[1] "double"
> length(x)
[1] 4
> y <- c('joe','bob','fred')
> typeof(y)
[1] "character"
> length(y)
[1] 3
> z <- c() # empty vector
> length(z)
[1] 0
> typeof(z)
[1] "NULL"
\end{R}

You can also use |c()| to concatenate two or more vectors together.
%
\begin{R}
> v <- c(1,3,5,7)
> w <- c(-1, -2, -3)
> vwx <- c(v,w,x)
> vwx
 [1]  1  3  5  7 -1 -2 -3  2  4  6  8
\end{R}

\subsubsection{Vector Arithmetic and Comparison}

The basic R arithmetic operations work on vectors as well as on
single numbers (in fact single numbers \emph{are} vectors).
%
\begin{R}
> x <- c(2, 4, 6, 8, 10)
> x * 2
[1]  4  8 12 16 20
> x * pi
[1]  6.283185 12.566371 18.849556 25.132741 31.415927
> y <- c(0, 1, 3, 5, 9)
> x + y
[1]  2  5  9 13 19
> x * y
[1]  0  4 18 40 90
> x/y
[1]      Inf 4.000000 2.000000 1.600000 1.111111
> z <- c(1, 4, 7, 11)
> x + z
[1]  3  8 13 19 11
Warning message:
longer object length
        is not a multiple of shorter object length in: x + z
\end{R}
%
When vectors are not of the same length R `recycles' the elements of the
shorter vector to make the lengths conform. In the example above
|z| was treated as if it was the vector |(1, 4, 7, 11, 1)|.


The comparison operators also work on vectors as shown below.
Comparisons involving vectors return vectors of booleans.
%
\begin{R}
> x > 5
[1] FALSE FALSE  TRUE  TRUE  TRUE
> x != 4
[1]  TRUE FALSE  TRUE  TRUE  TRUE
\end{R}

If you try and apply arithmetic operations to non-numeric vectors, R will warn you of the error of your ways:
%
\begin{R}
> w <- c('foo', 'bar', 'baz', 'qux')
> w**2
Error in w^2 : non-numeric argument to binary operator
\end{R}
%
Note, however that the comparison operators can work with non-numeric vectors. The results you get will depend on the type of the elements in the vector.
%
\begin{R}
>  w == 'bar'
[1] FALSE  TRUE FALSE FALSE
> w < 'cat'
[1] FALSE  TRUE  TRUE FALSE
\end{R}


\subsubsection{Indexing Vectors}

For a vector of length $n$, we can access the elements by the indices $1
\ldots n$. We say that R vectors (and other data structures like lists) are `one-indexed'. Many other programming languages, such as Python, C, and Java, use zero-indexing where the elements of a data structure are accessed by the indices $0 \ldots n-1$. Indexing errors are a common source of bugs. When moving back and forth between different programming languages keep the appropriate indexing straight!

Trying to access an element beyond these limits returns a special
constant called |NA| (Not Available) that indicates missing or non-existent
values.
%
\begin{R}
> x <- c(2, 4, 6, 8, 10)
> length(x)
[1] 5
> x[1]
[1] 2
> x[4]
[1] 8
> x[6]
[1] NA
> x[-1]
[1]  4  6  8 10
> x[c(3,5)]
[1]  6 10
\end{R}
%
Negative indices are used to exclude particular elements. |x[-1]| returns all
elements of |x| except the first. You can get multiple elements of a vector by
indexing by another vector. In the example above |x[c(3,5)]| returns the third
and fifth element of |x|.

\subsubsection{Combining Indexing and Comparison}

A very powerful feature of R is the ability to combine the comparison
operators with indexing. This facilitates data filtering and subsetting.
Some examples:
%
\begin{R}
> x <- c(2, 4, 6, 8, 10)
> x[x > 5]
[1]  6  8 10
> x[x < 4 | x > 6]
[1]  2  8 10
\end{R}
%

In the first example we retrieved all the elements of |x| that are larger than
5 (read as `x where x is greater than 5'). In the second example we retrieved
those elements of |x| that were smaller than four \emph{or} greater than six.
The symbol \lstinline!|! is the `logical or' operator. Other logical operators
include |&| (`logical and' or `intersection') and |!| (negation). Combining
indexing and comparison is a powerful concept and one you'll probably find
useful for analyzing your own data.

\subsubsection{Generating Regular Sequences}

Creating sequences of numbers that are separated by a specified value or
that follow a particular patterns turns out to be a common task in
programming. R has some built-in operators and functions to simplify
this task.
%
\begin{R}
> s <- 1:10
> s
 [1]  1  2  3  4  5  6  7  8  9 10
> s <- 10:1
> s
 [1] 10  9  8  7  6  5  4  3  2  1
> s <- seq(0.5,1.5,by=0.1)
> s
 [1] 0.5 0.6 0.7 0.8 0.9 1.0 1.1 1.2 1.3 1.4 1.5
# 'by' is the 3rd argument so you don't have to specify it
> s <- seq(0.5, 1.5, 0.33)
> s
[1] 0.50 0.83 1.16 1.49
\end{R}

|rep()| is another way to generate patterned data.
%
\begin{R}
> rep(c("Male","Female"),3)
[1] "Male"   "Female" "Male"   "Female" "Male"   "Female"
> rep(c(T,T, F),2)
[1]  TRUE  TRUE FALSE  TRUE  TRUE FALSE
\end{R}


\subsection{Some Useful Functions}

You've already seem a number of functions (|c()|, |length()|, |sin()|,
|log|, |length()|, etc). Functions are called by
invoking the function name followed by parentheses containing zero or
more \emph{arguments} to the function. Arguments can include the data
the function operates on as well as settings for function parameter
values. We'll discuss function arguments in greater detail below.

\subsubsection{Creating longer vectors}

For vectors of more than 10 or so elements it gets tiresome and error
prone to create vectors using |c()|. For medium length vectors
the |scan()| function is very useful.
%
\begin{R}
> test.scores <- scan()
1: 98 92 78 65 52 59 75 77 84 31 83 72 59 69 71 66
17:
Read 16 items
> test.scores
 [1] 98 92 78 65 52 59 75 77 84 31 83 72 59 69 71 66
\end{R}
%
When you invoke |scan()| without any arguments the function
will read in a list of values separated by white space (usually spaces
or tabs). Values are read until |scan()| encounters a blank
line or the end of file (EOF) signal (platform dependent). We'll see how to read in data from files below.

Note that we created a variable with the name |test.scores|.
If you have previous programming experience you might be surprised that
this works. Unlike most languages, R allows you to use periods in
variable names. Descriptive variable names generally improve readability
but they can also become cumbersome (e.g.
|my.long.and.obnoxious.variable.name|). As a general rule of
thumb use short variable names when working at the interpreter and more
descriptive variable names in functions.

\subsubsection{Useful Numerical Functions}

Let's introduce some additional numerical functions that are useful for
operating on vectors.
%
\begin{R}
> sum(test.scores)
[1] 1131
> min(test.scores)
[1] 31
> max(test.scores)
[1] 98
> range(test.scores) # min,max returned as a vec of len 2
[1] 31 98
> sorted.scores <- sort(test.scores)
> sorted.scores
 [1] 31 52 59 59 65 66 69 71 72 75 77 78 83 84 92 98
> w <- c(-1, 2, -3, 3)
> abs(w) # absolute value function
\end{R}

\subsection{Function Arguments in R}

Function arguments can specify the data that a function operates on or
parameters that the function uses. Some arguments are required, while
others are optional and are assigned default values if not specified.

Take for example the |log()| function. If you examine the help
file for the |log()| function (type |?log| now) you'll see that it takes two
arguments, refered to as `|x|' and `|base|'. The
argument |x| represents the numeric vector you pass to the
function and is a required argument (see what happens when you type
|log()| without giving an argument). The argument
|base| is optional. By default the value of |base|
is $e = 2.71828\ldots$. Therefore by default the |log()|
function returns natural logarithms. If you want logarithms to a
different base you can change the |base| argument as in the
following examples:
%
\begin{R}
> log(2) # log of 2, base e
[1] 0.6931472
> log(2,2) # log of 2, base 2
[1] 1
> log(2, 4) # log of 2, base 4
[1] 0.5
\end{R}

Because base 2 and base 10 logarithms are fairly commonly used, there are convenient aliases for calling log with these bases.
%
\begin{R}
> log2(8)
[1] 3
> log10(100)
[1] 2    
\end{R}

\subsection{Lists in R}

R lists are like vectors, but unlike a vector where all the elements are of the same type, the elements of a list can have arbitrary types (even other lists). 
\begin{R}
> l <- list('Bob', pi, 10, c(2,4,6,8))
\end{R}
Indexing of lists is different than indexing of vectors.  Double brackets (|x[[i]]|) return the element at index $i$, single bracket return a list containing the element at index $i$.
\begin{R}
> l[1] # single brackets
[[1]]
[1] "Bob"

> l[[1]] # double brackets
[1] "Bob"
> typeof(l[1])
[1] "list"
> typeof(l[[1]])
[1] "character"
\end{R}
The elements of a list can be given names, and those names objects can be accessed using the |$| operator. You can retrieve the names associated with a list using the |names()| function.
\begin{R}
> l <- list(name='Bob', age=27, years.in.school=10)
> l
$name
[1] "Bob"

$age
[1] 27

$years.in.school
[1] 10

> l$years.in.school
[1] 10
> l$name
[1] "Bob"
> names(l)
[1] "name"            "age"             "years.in.school"    
\end{R}


\subsection{Simple Input in R}

The \lstinline!c()! and \lstinline!scan()! functions are fine for
creating small to medium vectors at the interpreter, but eventually
you'll want to start manipulating larger collections of data. There are
a variety of functions in R for retrieving data from files.

The most convenient file format to work with are tab delimited text
files. Text files have the advantage that they are human readable and
are easily shared across different platforms. If you get in the habit of
archiving data as text files you'll never find yourself in a situation
where you're unable to retrieve important data because the binary data
format has changed between versions of a program.

\subsection{Using \lstinline!scan()! to input data}

\lstinline!scan()! itself can be used to read data out of a file.
Download the file \lstinline!algae.txt! from the class website and try
the following (after changing your working directory):

\begin{R}
> algae <- scan('algae.txt')
Read 12 items
> algae
 [1] 0.530 0.183 0.603 0.994 0.708 0.006 0.867 0.059 0.349 0.699 0.983 0.100
\end{R}
One of the things to be aware of when using \lstinline!scan()! is that
if the data type contained in the file can not be coerced to doubles
than you must specify the data type using the \lstinline!what! argument.
The \lstinline!what! argument is also used to enable the use of
\lstinline!scan()! with columnar data. Download \lstinline!algae2.txt!
and try the following:

\begin{R}
> algae.table <- scan('algae2.txt', what=list('',double(0)))
                        # note use of list argument to what
> algae.table

> algae.table
[[1]]
 [1] "Jan" "Feb" "Mar" "Apr" "May" "Jun" "Jul" "Aug" "Sep" "Oct" "Nov"
[12] "Dec"

[[2]]
 [1] 0.530 0.183 0.603 0.994 0.708 0.006 0.867 0.059 0.349 0.699 0.983
[12] 0.100

> algae.table[[1]]
 [1] "Jan" "Feb" "Mar" "Apr" "May" "Jun" "Jul" "Aug" "Sep" "Oct" "Nov"
[12] "Dec"
> algae.table[[2]]
 [1] 0.530 0.183 0.603 0.994 0.708 0.006 0.867 0.059 0.349 0.699 0.983
[12] 0.100
\end{R}
Use help to learn more about \lstinline!scan()!.

\subsection{Using \lstinline!read.table()! to input data}

\lstinline!read.table()! (and it's derivates - see the help file)
provides a more convenient interface for reading tabular data. Download the \lstinline|turtles.txt| data set from the class wiki.  The data in \lstinline|turtles.txt| are a set of linear measurements representing dimensions of the carapace (upper shell) of painted turtles (\textit{Chrysemys picta}), as reported in Jolicoeur and Mosimmann, 1960; Growth 24: 339-354.

Using the
file \lstinline!turtles.txt!:

\begin{R}
> turtles <- read.table('turtles.txt', header=T)
> turtles
   sex length width height
1    f     98    81     38
2    f    103    84     38
3    f    103    86     42
  # output truncated
> names(turtles)
[1] "sex"    "length" "width"  "height"
> length(turtles)
[1] 4
> length(turtles$sex)
[1] 48
\end{R}
What kind of data structure is \lstinline!turtles!? What happens when
you call the \lstinline!read.table()! function without specifying the
argument \lstinline!header=T!?

You'll be using the \lstinline!read.table()}!function frequently. Spend
some time reading the documentation and playing around with different
argument values (for example, try and figure out how to specify
different column names on input).

Note: \lstinline!read.table()! is more convenient but \lstinline!scan()!
is more efficient for large files. See the R documentation for more
info.

\subsection{Basic Statistical Functions in R}

There are a wealth of statistical functions built into R. Let's start to
put these to use.

If you wanted to know the mean carapace width of turtles in your sample
you could calculate this simply as follows:

\begin{R}
> sum(turtles$width)/length(turtles$width)
[1] 95.4375
\end{R}
Of course R has a built in \lstinline!mean()! function.

\begin{R}
mean(turtles$width) [1] 95.4375
\end{R}
One of the advantages of the built in \lstinline!mean()! function is
that it knows how to operate on lists as well as vectors:

\begin{R}
> mean(turtles)
      sex    length     width    height
       NA 124.68750  95.43750  46.33333
Warning message:
argument is not numeric or logical: returning NA in: mean.default(X[[1]], ...)
\end{R}
Can you figure out why the above produced a warning message? Let's take
a look at some more standard statistical functions:

\begin{R}
> min(turtles$width)
[1] 74
> max(turtles$width)
[1] 132
> range(turtles$width)
[1]  74 132
> median(turtles$width)
[1] 93
> summary(turtles$width)
   Min. 1st Qu.  Median    Mean 3rd Qu.    Max.
  74.00   86.00   93.00   95.44  102.00  132.00
> var(turtles$width) # variance
[1] 160.6769
> sd(turtles$width)  # standard deviation
[1] 12.67584
\end{R}


% \subsection{Simple Plots in R}

% One of the advantages of R is it's ability to produce a variety of plots
% and statistical graphics. Try out the following:

% \begin{R}
% > hist(turtles$width)  # histogram plot
% > hist(turtles$width,10) # produces a histogram with 10 bins
% > hist(turtles$width,breaks=10, xlab="Carapace Width", probability=T)
% > boxplot(turtles$width) # simple box plot
% # a fancy box plot showing multiple variables
% > boxplot(list(turtles$length, turtles$width,  turtles$height),
% +        names=c("Carapace\nLength","Carapace\nWidth","Carapace\nHeight"),
% +        ylab="millimeters")
% > title("Turtle Shell Variables")
% > plot(turtles$length, turtles$width)
% # how does this differ from the plot above?
% > plot(turtles$length ~ turtles$width)
% > plot(turtles$length, turtles$width,
% +      xlab="Carapace Length(mm)", ylab="Carapace Width(mm)")
% > title("Relationship Between\nLength and Width")
% \end{R}


% To get a sense of some of the graphical power of R try the
% \lstinline!demo()! function:
% %
% \begin{R}
% > demo(graphics)
% \end{R}
