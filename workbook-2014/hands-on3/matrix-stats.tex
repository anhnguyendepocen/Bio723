
\section{Descriptive statistics as matrix functions}

Assume you have a data set represented as a $n \times p$ matrix, $X$, with
observations in rows and variables in columns. Below I give formulae for
calculating some descriptive statistics as matrix functions.

\subsection{Mean vector and matrix}

You can calculate a row vector of means, $\mathbf{m}$, as: 
\[
\mathbf{m} = \frac{1}{n} \mathbf{1}^T  X
\] where $1$ is a $n \times 1$ vector of ones.

A $n \times p$ matrix $M$ where each column is filled with the mean
value for that column is: 
\[
M = \mathbf{1}\mathbf{m}
\]

\subsection{Deviation matrix}

To re-express each value as the deviation from the variable means
(i.e.~each columns is a mean centered vector) we calculate a deviation
matrix: 
\[
D = X - M
\]

\subsection{Covariance matrix}

The $p \times p$ covariance matrix can be expressed as a matrix product of the deviation matrix:
\[
S = \frac{1}{n-1} D^T D
\]

\subsection{Correlation matrix}

The correlation matrix, $R$, can be calculated from the covariance
matrix by: 
\[
R = V S V
\]

where $V$ is a $p \times p$ diagonal matrix where
$V_{ii} = 1/\sqrt{S_{ii}}$.

\subsection{Concentration matrix and Partial Correlations}

If the covariance matrix, $S$ is invertible, than inverse of the
covariance matrix, $S^{-1}$, is called the `concentration matrix' or
`precision matrix'. We can relate the concentration matrix to partial
correlations as follow. Let 
\[
P = S^{-1}
\]
Then:
\[
\mbox{corr}(x_i,x_j \mid X \backslash \{x_i,x_j\}) = -\frac{p_{ij}}{\sqrt{p_{ii} p_{jj}}}
\]

where $X \backslash \{x_i,x_j\}$ indicates all variables other than
$x_j$ and $x_i$. You can read this as `the correlation between x and y
conditional on all other variables.'

% \medskip

% \begin{assignment}
% Create an R library that includes functions that
% use matrix operations to calculate each of the descriptive statistics
% discussed above (except the concentration matrix / partial
% correlations). Calculate these statistics for \lstinline!iris! data set and check the results of your
% functions against the built-in R functions.
% \end{assignment}

