
\documentclass[11pt,letterpaper]{article}

\usepackage[T1]{fontenc}
\usepackage[sc]{mathpazo}

\usepackage{indentfirst}   %% if you want to have the first paragraph of each section indented %%
\usepackage[top=0.75in, bottom=0.75in, left=1in, right=1in]{geometry} %% for setting page margins %%
\usepackage{hyperref}
\hypersetup{
colorlinks=true
}
\usepackage{titlesec} % for more compact section spacing
\titlespacing{\section}{0pt}{*2}{*1}

\begin{document}


\section*{\centering Biology 723: Scientific Computing for Biologists, Fall 2013}
\begin{center}
\begin{tabular}{ll}
\textbf{Instructor}: Paul M. Magwene & \textbf{Office}: FFSC 4103\\
\textbf{Phone}: 613-8159 & \textbf{Email}: paul.magwene@duke.edu\\
&\\
\textbf{TA}: Colin Maxwell & \textbf{Office}: FFSC 3135B\\
\textbf{Phone}: 668-3755 & \textbf{Email}: colin.maxwell@duke.edu\\
\end{tabular}
\end{center}

\section*{Description}

The focus of this course is statistical computing for the biological sciences with an emphasis on common multivariate statistical methods and techniques for exploratory data analysis. A major goal of the course is to help graduate students in the biological sciences develop practical insights into methods that they are likely to encounter in their own research, and the potential advantages and pitfalls that come with their use.

\section*{Prerequisites}

Enrollment is limited to graduate students or undergraduates with permission of instructor. No previous programming experience is required, but familiarity with basic statistical concepts (equivalent of STA 213) is assumed.

\section*{Grading}
Grading is based on weekly homework assignments. These homework assignments will typically consist of statistical problem solving exercises and/or programming tasks.

\section*{Course Website}

\href{https://github.com/pmagwene/Bio723}{https://github.com/pmagwene/Bio723}

\renewcommand{\refname}{Texts}
\begin{thebibliography}{99}

\bibitem{Wickens} Wickens, T.\ D. 1995. The geometry of multivariate statistics. Lawrence Earlbaum Associates, New Jersey.

\bibitem{Matloff} Matloff, N. 2011. The Art of R Programming. No Starch Press, San Francisco.

\bibitem{CSpy}A. B. Downey, J. Elkner  and C.\ Meyers. How to think like a computer scientist: learning with Python. Available in \href{http://www.greenteapress.com/thinkpython/html/index.html}{HTML} and \href{http://www.greenteapress.com/thinkpython/thinkpython.pdf}{PDF} form under an open source license.



\end{thebibliography}

\renewcommand{\refname}{Other Recommended Texts}

\begin{thebibliography}{99}
\setcounter{enumiv}{5}
\bibitem{Hamilton} Hamilton, A.\ G. 1989. Linear algebra: An introduction with concurrent examples. Cambridge University Press, Cambridge.

A well organized and readable introduction to linear algebra. This assumes no previous familiarity with lineage algebra.  You'll get maximum benefit from this text if you work through the short exercises that accompany each chapter.

\bibitem{Krzanowski} Krzanowski, W.\ J. 2000. Principles of multivariate analysis. Oxford Univ. Press, New York.

I would have made this a required text but it's become unreasonably expensive, even for used copies. Nonetheless, if you plan on having just on book on multivariate statistics on your bookshelf this is the one I'd recommend.

\bibitem{Sokal} Sokal, R.\ R. and F.\ J. Rohlf. 1995. Biometry. W.\ H. Freeman, New York.

Another good text to have on your bookshelf. A readable and well organized basic statistics book with examples drawn from the biological literature.

\end{thebibliography}

\section*{Syllabus}

\renewcommand{\arraystretch}{1.4}
\begin{center}
\begin{tabular}{rp{5.5in}}
\multicolumn{1}{c}{{\sl Date}} & \multicolumn{1}{c}{{\sl Topic}} \\

August 27 & Introduction; Getting Acquainted with R \\
September 3 & Data as Vectors: Geometry of Correlation and Regression; Visualizing bivariate data in R\\
September 10 & Descriptive statistics as matrix operations; Visualizing and working with multivariate data in R\\
September 17 & Multiple regression and introduction to biplots; Regression in R\\
September 24 & Non-linear regression models\\
October 1 & Eigenvectors and Eigenvalues; Principal Components Analysis \\
October 8 & Singular Value Decomposition, Biplots, and Correspondence Analysis\\
October 15 & Discriminant analysis and Canonical Variate Analysis\\
October 22 & \multicolumn{1}{c}{{\sc Fall Break}} \\
October 29 & Analyses based on Similarity/Distance I; Hierarchical and K-means clustering\\
November 5 & Analyses based on Similarity/Distance II; Multidimensional scaling\\
November 12 & Randomization and Monte Carlo Methods; Jackknife, Bootstrap, Permutation\\
November 19 & Building Bioinformatics Pipelines I; Pipes, redirection, subprocesses \\
November 26 & Building Bioinformatics Pipelines II; Putting the concepts to work \\

& \multicolumn{1}{c}{{\sc Graduate Classes End November 26}} \\
\end{tabular}
\end{center}



\end{document}
